\documentclass[a4paper,icelandic]{article}

%% Use utf-8 encoding for foreign characters
\usepackage[T1]{fontenc}
\usepackage[utf8]{inputenc}
\usepackage{babel}

%% More symbols
\usepackage{amsmath}
\usepackage{amssymb}

%% Theorem
\usepackage{amsthm}
\theoremstyle{default}
\newtheorem{setn}{Setning}[section]
\theoremstyle{definition}
\newtheorem{skilgr}{Skilgreining}[section]
\theoremstyle{remark}
\newtheorem*{ath}{Athugasemd}

%% Kóði
\usepackage[scaled]{beramono}
\usepackage{listings}
\usepackage{listingsutf8}
\lstset{basicstyle=\footnotesize\ttfamily,
        breaklines=true,
        frame=single,
        inputencoding=utf8/latin1}

\title{%
\textbf{Greining reiknirita}\\
Verkefni 2\\
\emph{Kvik Huffman kóðun}}
\author{Hörður Freyr Yngvason}

\begin{document}
\maketitle

\section{Inngangur}
% Lýsum vandamáli sem við hyggjumst leysa
% Kynnum tvær tillögur að lausn í grófum dráttum
%  * þekkt viðmiðunarlausn
%  * ný lausn, hugsanlega betri en viðmiðunarlausnin
Segjum sem svo, að við viljum senda stafræn gögn yfir tengingu með
takmarkaða bandbreidd. Við viljum vitaskuld lágmarka flutningstímann,
sem jafngildir því að senda gögnin í eins fáum bitum og
unnt er. M.ö.o. viljum við \emph{þjappa} gögnunum á sem hagvkæmastan
hátt.

Ein vel þekkt þjöppunaraðferð er (upprunalega) \emph{kyrrlega
Huffman-kóðunin} \cite{mitAlgo}. Þar lítum við á skrá sem streng yfir
stafróf
ólíku stafanna í skránni, greinum tíðni hvers stafs og þjöppum
byggt á þeim upplýsingum. Farnar eru tvær umferðir
gegnum skrána. 
Fyrri umferðin fer í tíðnigreiningu, sem er notuð til að smíða svokallað
\emph{Huffman-tré} fyrir gögnin. Huffman-tréð má nota til að búa til
\emph{prefix-kóða} stafrófsins og lesa úr þjöppuninni.
Seinni umferðin fer í að þjappa gögnunum. 
Þessi aðferð hentar þó ekki fyrir allar gagnagerðir: Reikniritið krefst
þess að sjá öll gögnin áður en nokkru er þjappað, sem gengur ekki fyrir
gögn enn í smíði, t.d. beina útsendingu, þar sem næstu stafir eru
óþekktir. Í því tilfelli getum við valið um að seinka
útsendingunni og þjappa skránni í bútum, eða hverfa frá því að nota
kyrrlegu Huffman kóðunina. Einnig þarf að senda Huffman-tréð með boðunum
svo hægt sé að lesa þau.

Oft má gera betur; nefnilega, með því að fara aðeins eina umferð í
gegnum skrána og þjappa jafnóðum, getum við lágmarkað biðtíma og e.t.v.
þjappað gögnunum hraðar. Ein slík aðferð byggir á Huffman-kóðun, köllum
hana \emph{kvika Huffman-kóðun} \cite{vitter87}. Hugmyndin er að á meðan
þjöppun standi geymi sendandi og viðtakandi báðir eintak af sama
Huffman-trénu. Tréð sé Huffman-tré fyrir þann hluta gagnanna, sem búið
er að senda. Þá er næsta staf þjappað samkvæmt kóðun þessa trés hjá
sendanda og viðtakandi notar sitt tré til að endurheimta stafinn. Síðan
uppfæra báðir aðilar trén sín til að endurspegla breytt ástand
skilaboðanna, þ.a. báðir hafi enn sama tréð. Með þessu móti er aðeins
farin ein ferð um gögnin og þurfum ekki að senda tréð sérstaklega.
Meðal útfærsla á þessari hugmynd eru \emph{reiknirit FGK} (Faller,
Gallager, Knuth) \cite{knuth85}, síðar endurbætt af \emph{reikniriti
Vitter} \cite{vitter89}.

Í næsta hluta skoðum við þessar hugmyndir formlega, berum þær saman
og útskýrum í hvaða skilningi hvor fyrir sig er \emph{ákjósanleg}. Í
hluta \ref{sec:maelingar} birtum við niðurstöður samanburðar á
upphaflegu kyrrlegu Huffman-kóðuninni og kviku Huffman-kóðun Vitters
fyrir ýmis gögn. Öll forrit eru gefin í heild sinni aftast.



\section{Kyrrleg og kvik Huffman-kóðun}

Byrjum á að samræma rithátt fyrir þennan hluta
Látum $n$ vera stærð stafrófsins og $a_j$ vera $j$-ta staf þess,
$j=1\dots,n$. Látum $t$ vera fjölda stafa úr skilaboðunum, sem búið er
að vinna með, látum $M_t := (a_{i_j})_{j=1}^{j=t}$ vera runu
þeirra og $k$ vera fjölda ólíkra stafa í $M_t$. Fyrir öll $a_j$
látum við svo $w_j$ vera tíðni (þyngd) $a_j$  í $M_t$ og
$l_j$ vera fjarlægð laufs $a_j$ frá rót Huffman-trés $M_t$.
Köllum bitafjölda þjöppunar $M_t$ \emph{kostnað}
boðanna.

\subsection{Kyrrleg Huffman-kóðun}
Í fyrri umferð reikniritsins úthlutum við $k$ ólíku stökunum 
$a_j\in M_t$ hnút með gildið $a_j$ og þyngd $w_j$; setjum
hnútana í forgangsbiðröð $Q$ þannig að léttustu stökin fái hæstan
forgang. Þar til $Q$ inniheldur aðeins eitt stak, fjarlægjum við
ítrekað tvo léttustu hnútana $x$ og $y$ úr $Q$ og setjum í staðinn hnút
$z$, sem hefur $x$ og $y$ sem börn og samanlagaða þyngd þeirra. Þá
verður eina stakið eftir í $Q$ rót Huffman-trés fyrir $M_t$. Kóðunin
fyrir staf $a_j$ fæst þá með því að rekja sig frá rót niður í $a_j$
þannig að fyrir hverja stiku bætum við $0$ aftan á kóðann ef stikan er
til vinstri, en $1$ annars. 

Þau tré, sem fengist geta á þennan hátt (fyrir eitthvert mál)
kallast \emph{Huffman-tré}. Almennt, ef við smíðum einhvern prefix-kóða
fyrir stafina $a_j$ í $M_t$ og lítum á tilsvarandi tré, þá er stærðin 
$\sum_{j} w_j l_j$ kostnaður skilaboðanna $M_t$. 
Huffman-kóðun skilaboðanna er \emph{ákjósanleg}, í þeim skilningi að
kostnaður kóðunarinnar er lágmarkaður yfir öll prefix-kóðunartré fyrir
þetta sama stafróf.

Keyrslutími útfærslunnar hér að aftan er
$O(n\log(n))$, en með því að nota van Emde Boas tré á að vera hægt að ná
honum niður í $O(n\log(\log(n)))$ \cite{mitAlgo}.

\subsection{Kvik Huffman-kóðun}
%Lýsa í skilgreindu táknmáli hugmyndinni sem við lýstum óformlega í innganginum.

Í kvikri Huffman-kóðun notum við kvik Huffman-tré, þ.e. Huffman-tréð
okkar breytist og lagar sig að skilaboðunum gegnum kóðunina.
$(t+1)$-ti stafurinn, $a_{i_{t+1}}$, er kóðaður út frá Huffman-trénu
fyrir $M_t$; sendandi og viðtakandi breyta síðan báðir Huffman-trénu
sínu í samræmi við viðbótina þannig að báðir enda með sama Huffman-tréð
fyrir $M_{t+1}$. Vandinn er þá fyrst og fremst að framkvæma þetta skref
skilvirkt; sér í lagi þarf að tryggja að tréð sé enn Huffman-tré. Til
þess notfærum við okkur \emph{systkinaeiginleika} Huffman-trjáa:
Tvíundatré með þyngd $\ge 0$ og $p$ lauf er Huffman-tré \emph{þ.þ.a.a.}
það uppfylli eftirfarandi tvö skilyrði \cite{vitter87}
\begin{enumerate}
  \item laufin $p$ hafa þyngdir $w_1,\dots,w_p\geq 0$ og þyngd innri
    hnúts er summa barna hans;
  \item unnt er að númera hnútana í vaxandi röð eftir þyngd, þannig að
    hnútar $2j-1$ og $2j$ eru systkini, fyrir $j=1,\dots,p-1$, og
    sameiginelga foreldri þeirra er af hærra númeri.
\end{enumerate}
Í þessu liggur kjarni reiknirits FGK, sem er síðan endurbættur í
reikniriti Vitters: Við viljum framkvæma skrefið $H_t\mapsto H_{t+1}$,
þar sem $H_i$ er Huffman-tré $M_i$. Framkvæmum eitt milliskref á $H_t$:
Byrjum með hnút, sem er laufið fyrir $a_{i_{t+1}}$. Skiptum ítrekað á
innihaldi hnútsins, þar með talið er undirtré hans, og innihaldi hæst
númeraða hnútnum með sömu þyngd, skoðum svo í staðinn foreldri
síðarnefnda hnútsins. Að þessu loknu verður $H_t$ enn Huffman-tré fyrir
$M_t$ og við fáum $H_{t+1}$ með því að hækka þyngd laufs
$a_{i_{t+1}}$ og forfeðra þess um $1$ (því við höfum varðveitt
systkinaeiginleikann).

Athugum sérstaklega tilfellið $k < n$, þ.e. þegar við höfum enn ekki séð
allt stafrófið. Þá notum við einn $0$-hnút til að tákna þá $n-k$ stafi
sem við eigum eftir að sjá. Ef $a_{i_{t+1}}$ er slíkur stafur, þ.e.
$a_{i_{t+1}}\notin M_t$, þá kljúfum við $0$-hnútinn og smíðum úr öðrum
hlutanum laufið fyrir $a_{i_{t+1}}$. Kóðinn sem við sendum fyrir
$a_{i_{t+1}}$ er kóðinn frá rót niður í $0$-hnútinn, ásamt nokkrum
aukabitum sem segja, hver $n-k$ ónotuðu stafanna var sendur.

\subsubsection{Reiknirit Vitter}

Reiknirit Vitters er svipað, nema hvað við númerum hnútana út frá
myndframsetningu trésins, köllum það \emph{óbeina} númeringu: Hnútarnir
eru númeraðir í vaxandi röð eftir dýpt, en hnútar af sömu dýpt eru
númeraðir í vaxandi röð frá vinstri til hægri. Við viðhöldum þessari
númeringu með sérstakri gagnagrind sem við köllum \emph{fleytitré} (e.
floating tree). Við höfum jafnframt eftirfarandi fastayrðingu
\begin{quote}
  Látum $w$ vera þyngd í trénu. Þá eru öll lauf af þyngd $w$
  á undan öllum innri hnútum af þyngd $w$ með tilliti til óbeinu
  númeringarinnar.
\end{quote}
Sérhvert Huffman-tré sem uppfyllir þetta skilyrði lágmarkar bæði
$\sum_{j} l_j$ og $\max_j\{l_j\}$. Að lokum, til að einfalda útfærslu,
skilgreinum við:
\begin{skilgr}
  \emph{Kubbar} trésins (e. blocks) eru jafngildisflokkar venslanna
  \begin{quote}
    $x\equiv y$ e.f.f. $x$ og $y$ hafa sömu þyngd og eru báðir innri
    hnútar eða báðir lauf.
  \end{quote}
  \emph{Leiðtogi} (e. leader) kubbs er sá hnútur kubbsins, sem hefur
  hæstu óbeinu númeringuna.
\end{skilgr}

\subsection{Samanburður á aðferðunum}
Erfitt er að fá áþreifanlegt mat á gæðum þjöppunarreiknirits út af fyrir
sig, því vel má hugsa sér sérhæfð mynstur sem hægt er að þjappa mun
betur en með Huffman-legri kóðun; t.d. er segðin
\begin{quote}
  fyrstu $10^6$ aukastafir $\pi$ í tugakerfi
\end{quote}
greinileg þjöppun á útlistun umræddra aukastafa á pappír. Við einskorðum
okkur því hér við að meta kviku kóðunina út frá kyrrlegu kóðuninni.
Táknum með $S_t$ kostnað kyrrlegu kóðunarinnar og látum $D_t$ standa
fyrir kostnað kviku kóðunar Vitters (fyrir $M_t$). Nánar tiltekið er
kostnaðurinn fjöldi bita, sem sendir eru, nema hvað
\begin{itemize}
  \item í kyrrlegu kóðuninni teljum við ekki með kostnaðinn við að senda
    tréð,
  \item í kviku kóðuninni teljum við ekki aukabitana sem sendir eru,
    þegar við sjáum staf í fyrsta skipti.
\end{itemize}
Í grein sinni \cite{vitter87} sýndi Vitter fram á eftirfarandi
meginskorður:
\begin{setn}\label{setn:diff}
  \begin{enumerate}
    \item $S_t - k \leq D_t \leq S_t + t$
    \item Reiknirit Vitter lágmarkar muninn $D_t - S_t$ yfir öll $M_t$
      og öll (einnar umferðar) kvik Huffman-reiknirit. 
  \end{enumerate}
\end{setn}

\section{Niðurstöður mælinga}\label{sec:maelingar}
Mælingar aðallega gerðar á stafrófi af stærð 256.

\section{Lokaorð}
Taka saman lykilatriði í samanburðinum á aðferðunum.

Athuga að kvika kóðunin er mjög veik fyrir gagnatapi.
Athuga að fyrir kviku kóðunina þarf stafrófið að vera þekkt fyrir kóðun!
Það er frekar stór krafa.

\section*{Viðauki: Forritin sem notuð voru í mælingarnar}

Huffman-kóðunin er útfærð með \cite{mitAlgo} til hliðsjónar.
Keyrslutíminn ætti að vera $O(n\log n)$, en forritið er takmarkað við að
gögnin komist fyrir í minninu. Reiknirit Vitters er útfært í klasanum
FloatingTree og er svo gott sem Pascal$\to$C++ þýðing á \cite{vitter89}.
Notkunin er 
\begin{lstlisting}
  ./huffman [t]
  ./vitter [t]
\end{lstlisting}
þar sem \texttt{t} er fjöldi stafa í skránni (-1 og ekkert þýðir þjappa
öllu); gögnin eru lesin af aðalinntaki.

\lstinputlisting[language=C++,title=\lstname]{src/huffman.cpp}
\lstinputlisting[language=C++,title=\lstname]{src/vitter.cpp}


\begin{thebibliography}{9}
  \bibitem{mitAlgo} Cormen, T.H; Leiserson, C.E; Rivest, R.L; Stein, C.
    2009. Introduction to Algorithms, 3rd ed. 428-435.
  \bibitem{knuth85} Knuth, D.E. 1985. „Dynamic Huffman coding“.
    \emph{Journal of Algorithms 6}, 2, 163-180.
 \bibitem{vitter87} Vitter, J.S. 1987. „Design and Analysis of Dynamic
    Huffman Codes“. \emph{J. ACM 34}, 4, 825-845.
  \bibitem{vitter89} Vitter, J.S. 1989. „Algorithm 673: Dynamic Huffman
    Coding“.  \emph{ACM Trans. Math. Softw. 15}, 2, 158-167.
\end{thebibliography}
\end{document}
