\documentclass[a4paper,icelandic]{article}
\usepackage[T1]{fontenc}
\usepackage[utf8]{inputenc}
\usepackage{babel}

\usepackage[unicode=true, pdfusetitle,
 bookmarks=true,bookmarksnumbered=false,bookmarksopen=false,
 breaklinks=false,pdfborder={0 0 1},backref=false,colorlinks=false]
 {hyperref}

\usepackage{amsmath}

\usepackage{enumerate}

% Fyrir mállýsingar
\usepackage[rounded]{syntax}
%\grammarindent=5.3em

% Fyrir kóða
\usepackage{verbatim}

\title{Handbók fyrir Schmorpho}
\author{Hörður Freyr Yngvason}

\begin{document}
\maketitle

\section{Inngangur}
Fyrir áfangann Þýðendur (TÖL202M), vorönn 2010, átti hanna
forritunarmál og búa til þýðanda frá málinu yfir í Morpho
þulu. Afraksturinn var málið \emph{Schmorpho}, (nánast) hlutmengi af
\textbf{Sch}eme í \textbf{Morpho}.

Morpho þulan er forritunarmál m.a. sniðið í kring um vakningarfærslur til
að keyra á Morpho vélinni \cite{morphodist}. Innbyggður stuðningur er
við halaendurkvæmni og lokanir, sem gerir það að verkum að vélin er
kjörin í keyrsluhluta fallsforritunarmáls. Markmið mitt í verkefninu
var fyrst og fremst að \emph{eyða sem minnstri vinnu í þýðandann, en
  fá um leið sem öflugast forritunarmál}. Það ætti því ekki að koma á
óvart að stefnan hafi verið tekin á hlutmengi í Scheme. Nánar tiltekið
einskorðaði ég mig við lykilorðin \verb|if| og \verb|define| en reyndi
á móti að útfæra þau sem best. 

\subsection{Dæmi um forrit}
\label{sec:dami-um-forrit}

Eftirfarandi kóða má þýða í keyrslueiningu með aðalfall
\verb|main|. Hjálparföll á borð við \verb|+|, \verb|++|,
\verb|print| og \verb|%| eru flutt inn úr \verb|BASIS|
einingunni í Morpho. 

\verbatiminput{examples/fibgcd.schmo}

\section{Notkun og uppsetning}
Fyrir utan Morpho útgáfuna \cite{morphodist} samanstendur
  Schmorpho af skránum í töflu \ref{tab:skrar}.
\begin{table}[h!]
  \centering
  \begin{tabular}{lp{0.7\linewidth}}
    \hline 
    \textbf{Skrá}        & \textbf{Hlutverk}\\\hline\hline 
    \verb|lesgrein.l|    & Lesgreinir skrifaður í flex\cite{flex}    \\\hline 
    \verb|thattari.y|    & þáttari skrifaður í Bison \cite{bison}    \\\hline 
    \verb|CodeGen.h/cpp| & Klasasafn fyrir milliþulu\\\hline 
    \verb|schmorpho-asm| & þulusmiðurinn. þarf að þýða sérstaklega.  \\\hline 
    \verb|schmorpho|     & Einföld, stillanleg Bash-skripta sem tengir saman
                           Schmorpho þulusmiðinn og Morpho þýðandann \\\hline 
    \verb|makefile|      & \verb|make|-skrá sem notar GCC, flex og Bison
                           \cite{gcc,flex,bison} til að framleiða þulusmiðinn,
                           þ.e. forritið \verb|schmorpho-asm|        \\\hline
  \end{tabular}
  \caption{Lykilskrárnar í Schmorpho}
  \label{tab:skrar}
\end{table}
Til að þýða þulusmiðinn má nota nýlega útgáfu af GCC, Flex og Bison,
síðan nægir að nota \verb|make|-skrána í aðalmöppunni. Þá fæst
keyrsluskráin \verb|schmorpho-asm| sem er þulusmiðurinn fyrir
Schmorpho.  

\subsection{Þulusmiðurinn \texttt{schmorpho-asm}}
\label{sec:notkun-pulusmidarins}

Þulusmiðurinn les Schmorpho-kóða af aðalinntaki og skrifar þulu á
aðalúttak. Hægt er að þýða Schmorpho forritsskrá \verb|skrá.schmo| á
þrjá vegu
\begin{enumerate}
\item
\begin{verbatim}
schmorpho-asm < skrá.schmo
\end{verbatim}
  Skrifar hráa þulu fyrir föllin í skránni án þess að setja þau í einingu.
\item
\begin{verbatim}
schmorpho-asm nafn < skrá.schmo
\end{verbatim}
  Skrifar þulu fyrir hringtengda einingu með nafnið \verb|nafn.mmod|
  og flytur inn \verb|BASIS| en er án keyrslufalls.
\item
\begin{verbatim}
schmorpho-asm nafn aðalfall < skrá.schmo
\end{verbatim}
  Skrifar þulu fyrir hringtengda keyrslueiningu sem flytur inn
  \verb|BASIS|, hefur nafnið \verb|nafn.mexe| og keyrslufallið
  \verb|aðalfall|, sem verður að vera til staðar í \verb|skrá.schmo|.
\end{enumerate}

\section{Málfræði}
\label{sec:malfradi}

\subsection{Frumeiningar málsins}
\label{sec:frumeiningar-malsins}

\subsubsection{Sértákn}
\label{sec:sertakn}
Einu sértáknin eru svigar. Við notum aðeins venjulega sviga en tökum
líka frá hornklofa og slaufusviga til að fyrirbyggja ruglingsleg nöfn.

\subsubsection{Lykilorð}
\label{sec:lykilord}
Í málinu eru aðeins tvö lykilorð, \verb|define| og \verb|if|. Við
notum \verb|define| til að skilgreina öll \emph{nöfn}, en \verb|if| er
notað til að tákna val á milli tveggja ólíkra segða byggt á niðurstöðu
samanburðarsegðar.

\subsubsection{Athugasemdir}
\label{sec:athugasemdir}
Allt fyrir aftan \verb|;| í línu telst vera athugasemd. Athugasemdir
eru hunsaðar við þýðingu.

\subsubsection{Nöfn}
\label{sec:nofn}
Nöfn \synt{name} eru notuð til að vísa í föll og önnur gildi. Þau mega
samanstanda af ýmis konar sértáknum, bókstöfum og tölustöfum, en mega
ekki hefjast á tölustaf.
\begin{grammar}
  <name> ::=
  \begin{syntdiag*}
    \begin{stack}
      <alpha>\\
      <opchar>
    \end{stack}
    \begin{rep} \\
      \begin{stack}
        <alpha>\\
        <opchar> \\
        <digit>
      \end{stack}
    \end{rep}
  \end{syntdiag*}

  <alpha> ::= allir íslenskir stafir

  <opchar> ::= `+' | `-' | `*' | `/' | `\%' | `\$' | `!' | `_' | `:' |
  `|' \\ | `<' | `=' | `>' | `^' | `#' | `\&' | `?' | `@'
\end{grammar}

\subsubsection{Frumstæð gildi}
\label{sec:frumstad-gildi-1}
Hugmyndin er að nota nákvæmlega sömu frumstæðu gildin og Morpho
þulan. Það eru Bool-gildi \synt{bool}, heiltölur \synt{integer},
fleytitölur \synt{float}, bókstafir \synt{char} og strengir
\synt{string} og \lit{null}.  Táknum öll frumstæð gildi einu nafni
\synt{value}. Bool-gildi eru \lit{true} og \lit{false}; fleytitölur
eru í tugakerfi, strengir eru safn stafa afmarkað af tvöföldum
gæsalöppum og bókstafir eru stakir stafir innan einfaldra gæsalappa;
\lit{null} er tóma tilvísunin í Morpho.  Nánar tiltekið höfum við
\begin{grammar}
  <value> ::= <bool> | <integer> | <float> | <char> | <string> |
  \lit{null}

  <bool> ::= \lit{true} | \lit{false}
  
  <integer> ::=
  \begin{syntdiag*} \begin{rep} <digit> \end{rep} \end{syntdiag*}

  <digit> ::= 0 | 1 | 2 | 3 | 5 | 6 | 7 | 8 | 9

  <float> ::=
  \begin{syntdiag*}
    \begin{stack}
      <integer> `.'
      \\
      <integer> `.' <integer>
      \\
      `.' <integer>
    \end{stack}
  \end{syntdiag*}

  <char> ::= \begin{syntdiag*}`\'' <char\_v> `\''\end{syntdiag*}

  <string> ::=
  \begin{syntdiag*}
    `"'
    \begin{rep}
      <char\_v>
    \end{rep}
    `"'
  \end{syntdiag*}

  <char\_v> ::=
  \begin{syntdiag*}
    \begin{stack}
      \tok{allir stafir nema `\''}\\
      `\\' `\'' \\
      `\\' `\\' \\
      `\\' `n' \\
      `\\' `r' \\
      `\\' `t' \\
      `\\' `b' \\
      `\\' `f' \\
      `\\' <octal> \\
      `\\' <octal> <octal> \\
      `\\'
      \begin{stack}
        `0' \\
        `1' \\
        `2' \\
        `3'
      \end{stack}
      <octal> <octal>
    \end{stack}
  \end{syntdiag*}

  <octal> ::= 0 | 1 | 2 | 3 | 4 | 5 | 6 | 7
\end{grammar}

\subsection{Mállýsing}
\label{sec:mallysing}

Hér má sjá sömu mállýsingu og þáttarinn notar, nema á
hreinu BNF-formi og án hliðarverkana sem þulusmíðin felur í sér.
\begin{grammar}
  <program> ::= $\epsilon$ | <program> <defun>
  
  <defun> ::= `(' `define' `(' <name> <args> `)' <body> `)'
  
  <defvar> ::= `(' `define' <name> <expr> `)'
  
  <args> ::= $\epsilon$ | <args> <name>
  
  <body> ::= <expr>
  \alt <defuns> <expr>
  \alt <defvars> <expr>
  \alt <defuns> <defvars> <expr>
  
  <defuns> ::= <defun> | <defuns> <defun>
  
  <defvars> ::= <defvar> | <defvars> <defvar>
  
  <exprs> ::= $\epsilon$ | <exprs> <expr>
  
  <expr> ::= <cexrp> | <vexpr>
  
  <cexpr> ::= `(' <name> <exprs> `)'
  \alt `(' <cexpr> <exprs> `)'
  \alt `(' `if' <expr> <expr> <expr> `)'
  
  <vexpr> ::= <name> | <value>
\end{grammar}

\section{Merking málsins}
\label{sec:merking-malsins}

\subsection{Schmorpho forrit}
\label{sec:schmorpho-forrit}

Schmorpho forrit, \synt{program}, er runa skilgreininga á föllum,
\synt{defun}, sem mynda saman eina einingu í Morpho. Venjulega látum
við eininguna vera hringtengda til að líkja eftir virkni Scheme, en
ef engin viðföng eru send inn í þulusmiðinn fæst einfaldlega runa
falla, sem er ekki sett inn í einingu (þ.e. ekki umlukin tvöföldum
slaufusvigum). 

\subsection{Skilgreiningar}
\label{sec:skilgreiningar}

Skilgreiningar í mállýsingunni eru táknaðar með \synt{defun} fyrir
föll og \synt{defvar} fyrir breytur.
Skilgreining er ekki gildi, heldur gefur hún gildi eða útfluttu
falli nafn. Skilgreiningin \synt{defun} í \synt{program} er
t.d. skilgreining á útfluttu falli, en \synt{defun}  í stofni falls,
\synt{body}, skilgreinir innra fall (lokun) í foreldri sínu.

\subsubsection{Skilgreiningar á föllum}
\label{sec:skilgr-foll}

Föll eru skilgreind með
\begin{grammar}
  <defun> ::= `(' `define' `(' <name> <args> `)' <body> `)'
\end{grammar}
hvort sem þau standa fyrir innra eða ytra fall. Munurinn á innri og
ytri föllum er sá, að nöfn innri falla eru líka breytur með gildi, sem
er lokunin fyrir fallið, á meðan nöfn ytri falla standa ekki fyrir
nein gildi (sjá nánar í hluta \ref{sec:gildi}).  Í skilgreiningu falls
er \synt{name} nafn þess, \synt{args} listi nafna viðfanga þess (núll
eða fleiri) og \synt{body} stofn þess. Stofninn er þannig, að fyrst
koma skilgreiningar allra innri falla, síðan allar
breytuskilgreiningar, og loks ein segð, sem er skilagildi fallsins,
þ.e.
\begin{grammar}
  <body> ::=
  \begin{syntdiag*}
    \begin{rep}
      \\
      <defun>
    \end{rep}
    \begin{rep}
      \\
      <defvar>
    \end{rep}
    <expr>
  \end{syntdiag*}
\end{grammar}
Jafngild, en aðeins flóknari lýsing er í mállýsingunni
(\ref{sec:mallysing}), þar eð taka þurfti tillit til viðvarana frá
Bison um árekstra. 

\subsubsection{Breytuskilgreiningar}
\label{sec:breytuskilgreiningar}
Skilgreiningar af gerðinni
\begin{grammar}
  <defvar> ::= `(' `define' <name> <expr> `)'
\end{grammar}
skilgreina breytunafnið \synt{name} þannig að það standi fyrir gildi
segðarinnar \synt{expr}. Þær geta ekki verið endurkvæmar.

\subsection{Sýnileiki nafna}
\label{sec:synileiki-nafna}

Innri föll eru líka kölluð \emph{földuð föll} og fjöldi skilgreininga
utan um faldað fall er \emph{földunarhæð þess}. Í skilgreiningu falls
lítum við svo á, að viðföngin \synt{args} og innihald stofnsins
\synt{body} séu á sömu földunarhæð og einni földunarhæð fyrir neðan
nafn fallsins, \synt{name}. Ystu föll eru á földunarhæð 0 og við lítum
svo á að þar séu engin nöfn. Við leyfum skilgreiningu nafns, svo lengi
sem það er ekki þegar til á földunarhæð sinni, en nöfn eru tekin frá í
þeirri röð sem þau koma fyrir á hverri földunarhæð. Aftur á móti er
leyfilegt að endurskilgreina nafn af ytri földunarhæð, en þá er engin
leið að nálgast það aftur á þeirri og innri hæðum. \footnote{Reyndar
  er ekkert gert til að fyrirbyggja tvískilgreiningu nafns á sömu
  földunarhæð. Tilvísunin týnist í breytutöflu \emph{þýðandans}, en
  gildið lifir enn í tilheyrandi vakningarfærslu og \emph{lekur minni},
  þar sem ruslasafnarinn mun halda að það sé í notkun lengur en þarf.}

Örlítið flóknari (en afskaplega eðlilegar) reglur gilda, um notkun
nafna. Í hvert sinn sem við sjáum vísað í nafn, þá lítum við svo á að
nafnið standi fyrir \emph{nálægustu sýnilegu skilgreiningu nafnsins}.
\begin{enumerate}
\item Við sjáum ekkert af innri hæðum.
\item Nöfn viðfanga, þ.e. \synt{args}, eru sýnileg alls staðar á sinni
  földunarhæð
\item Öll föll, sem skilgreind eru á sömu földunarhæð með
  skilgreiningu af gerðinni \synt{defun} eru sýnileg alls staðar á
  sinni földunarhæð, sér í lagi sín á milli.
\item Aðrar breytur, þ.e. allt skilgrent í \synt{defvar} hluta stofns,
  eru sýnilegar í þeirri röð sem þær eru skilgreindar en ósýnilegar
  sjálfum sér og öllu á undan sér.
\item Ef nafn er sýnilegt á földunarhæð tilvísunarinnar með tilliti
  til uppruna hennar (úrskurðað með 1-4), þá er sú skilgreining
  nálægust. Annars lítum við á umlykjandi fall sem \emph{uppruna}
  tilvísunarinnar og leitum aftur á földunarhæð þess. Ef við lendum á
  földunarhæð 0, þá þýðir það að nafnið er innflutt og verður að vera
  notað sem fall (þ.e. ekki gildi).
\end{enumerate}
Reglur 3 og 5 gefa saman að föll á sömu földunarhæð geta verið
innbyrðis endurkvæm, sér í lagi getur fall kallað endurkvæmt á sjálft
sig. Aftur á móti geta breytuskilgreiningar af gerðinni \synt{defvar}
ekki verið endurkvæmar.



\subsection{Gildi og segðir}
\label{sec:gildi}
Nú þegar við höfum rætt um skilgreiningar, þá er nauðsynlegt að vita
hvað telst til gilda í málinu. Gildi og segðir eru sami hluturinn í
Schmorpho. Þ.e. allar segðir eru gildi, og öll gildi eru segðir. Við
táknum segðir með \synt{expr} í mállýsingunni og skiptum þeim upp í
tvo flokka, \synt{vexpr} og \synt{cexpr}. Við getum hugsað okkur að
\synt{cexpr} séu segðir sem þarf að reikna og \synt{vexpr} séu gildi
sem búið er að reikna. Meginástæðan fyrir þessari flokkun er þó sú, að
kall á fall getur verið tvenns konar, þ.e. kall á innflutt fall eða
kall á lokun. Það þýðir að fyrir kall af gerðinni
\begin{syntdiag*}
  `(' <name> <exprs> `)' 
\end{syntdiag*}
þurfum við að kanna hvort \synt{name} er innflutt nafn. Með flokkuninni
á \synt{expr} getum við merkt þessi vafatilfelli strax í þáttun, sem
getur einfaldað þýðingu til muna.

\subsubsection{Frumstæð gildi}
\label{sec:frumstad-gildi}
Frumstæð gildi \synt{value}, þ.e. bool-gildi, heiltölur, fleytitölur, strengir og
stafir ásamt \verb|null| eru sér í lagi gildi/segðir. Sjá nánar í hluta
\ref{sec:frumeiningar-malsins}.

\subsubsection{Breytur}
\label{sec:breytur-3}
Breytur eru nöfn, \synt{name}, sem standa fyrir gildi. Allar breytur eru nöfn
(sjá hluta \ref{sec:frumeiningar-malsins}), en sum nöfn eru ekki
breytur (sjá nánar \ref{sec:koll-foll} og \ref{sec:lokanir}). Þótt við
tölum um \emph{breytur} þá bjóðum við ekki sérstaklega upp á neinar
aðgerðir til að breyta gildinu sem breytunöfn standa fyrir. Mælst er til þess að
forrita sem mest án slíkra hliðarverkana, en það þó ekki bannað og auðvelt að
flytja inn föll sem valda hliðarverkunum.

\subsubsection{If-segðir}
\label{sec:if-segdir}

If-segðir í Schmorpho eru á forminu
\begin{grammar}
  <ifexpr> ::= `(' `if' <cond> <then> <else> `)'
\end{grammar}
þar sem \synt{cond}, \synt{then} og \synt{else} eru
\synt{expr}. Segðin \synt{cond} er ákvarðanatökusegð: Ef útkoma hennar
er önnur en \verb|false| eða \verb|null|, þá fær \synt{ifexpr} gildið
\synt{then} en annars \synt{else}.

\subsubsection{Köll á föll}
\label{sec:koll-foll}
Köll á föll gefa alltaf eitthvert gildi, þótt útkoman geti verið
óskilgreind; t.d. látum við liggja á milli hluta hvert gildi
segðarinnar \verb|(print "Halló")| er. Köll á föll eru af gerðinni
\begin{grammar} <funcall> ::=
  \begin{syntdiag*} `('
    \begin{stack} <name> \\ <cexpr>
    \end{stack} <exprs> `)'
  \end{syntdiag*}
\end{grammar}
þar sem \synt{name} er nafn fallsins (lokun eða innflutt),
\synt{cexpr} er lokun fyrir fallið og \synt{exprs} er runa þeirra
gilda sem fallið tekur sem viðföng. Þetta þýðir sér í lagi að
\emph{öll viðföng eru gildisviðföng}. Við segjum að nafn eða fall sé
\emph{innflutt} ef það er notað eins og fall, en er hvergi skilgreint
þar sem kallið getur \emph{séð} það. Til eru óleyfileg köll, sem þó
uppfylla málritið, t.d. getur útkoman úr \synt{cexpr} verið \verb|1|,
sem er ekki fall. Slíkar villur koma ekki fram fyrr en hugsanlega í
Morpho-hluta þýðingar/keyrslu.

\paragraph{Halaendurkvæmni}
\label{sec:halaendurkvamni}

Seinasta segðin í stofni falls er alltaf skilagildi þess, svo ef hún
er kall á fall, þá má það kall vera halaendurkvæmt. Þetta gefur okkur
skilvirka leið til að ítra án þess að nota lykkjur. Dæmi um þetta má
sjá í greinum \ref{sec:dami-um-forrit} og \ref{sec:fleiri-dami-um}.

\subsubsection{Lokanir}
\label{sec:lokanir}
Í Schmorpho eru öll innri föll lokanir, en \emph{innflutt föll eru það
  ekki}.  Það þýðir að ekki er hægt að senda innflutt fall sem viðfang
í annað fall, né láta innri breytu standa fyrir innflutt fall. Aftur á
móti er hægt að búa til lokun, sem kallar á innflutt fall og láta
innflutt fall skila lokun. T.d.
\begin{verbatim}
(define (kósínus)
  ;; Dæmi 1
  (define g cos)           ;; Rangt! cos er innflutt.
  (define (g x) (cos x))   ;; Rétt! g er lokun jafngild cos.
  ;; Dæmi 2
  (define (e fun) (fun x)) ;; Gildistökuvörpun
  (define h (e cos))       ;; Rangt! cos hefur ekki gildi
  (define h (e g))         ;; Rétt! g er lokun
  g)                       ;; skilum g, sem er lokun

;; Dæmi 3
(define (kósínus-af-0) 
  (define f (kósínus)) ;; gildið er lokun sem er jafngild cos
  (f 0))
\end{verbatim}

\subsubsection{ Hlutir og önnur gildi úr Morpho}
\label{sec:hlutir-og-onnur}
Schmorpho er langt því frá að nýta allar mögulegar aðgerðir Morpho
þulunnar. Til dæmis styður Morpho þulan hlutbundna forritun en ekki
Schmorpho. Hins vegar væri hægt að flytja inn slíka hluti og fela þá á
bak við föll, þannig að Schmorpho geti notið góðs af þeim. 

\section{Fleiri dæmi um forrit}
\label{sec:fleiri-dami-um}

\subsection{Listaföll eins og í Scheme}
\label{sec:listafoll-eins-og}

Í venjulega Schmorpho notum við föllin í \verb|BASIS| beint.  Hér
er dæmi um hvernig nota mætti aðeins eitt fall úr \verb|BASIS|, nefnilega
\verb|==|, til þess að útfæra listaföllin \verb|cons|, \verb|car|,
\verb|cdr|, \verb|map| og \verb|foldl|. Við erum örlítið takmörkuð af
því að hafa ekkert \verb|list| fall; í staðinn notum við \verb|cons|
ítrekað og samsömum \lit{null} við tóman lista.

\verbatiminput{examples/cons.schmo}



\begin{thebibliography}{4}
\bibitem{morphodist}Snorri Agnarsson. Morpho þýðandinn og keyrsluumhverfið.\\
  \href{http://morpho.cs.hi.is/}{http://morpho.cs.hi.is/}
\bibitem{bison}Bison - GNU parser generator. \\
  \href{http://www.gnu.org/software/bison/}{http://www.gnu.org/software/bison/}
\bibitem{flex}flex: The Fast Lexical Analyzer. \\
  \href{http://flex.sourceforge.net/}{http://flex.sourceforge.net/}
\bibitem{gcc}GCC, the GNU Compiler Collection.\\
  \href{http://gcc.gnu.org/}{http://gcc.gnu.org/}
\end{thebibliography}

\end{document}
