\documentclass[a4paper,icelandic]{article}

%% Use utf-8 encoding for foreign characters
\usepackage[T1]{fontenc}
\usepackage[utf8]{inputenc}
\usepackage{babel}

%% Vector based fonts instead of bitmaps
\usepackage{lmodern}

%% Useful
\usepackage{fullpage} % Smaller margins
\usepackage{enumerate}
\usepackage{verbatim}
\usepackage{hyperref}
\usepackage{graphicx}
\usepackage{multirow}

%% More math
\usepackage{amsmath}
\usepackage{amssymb}

%% Document Header
\title{\textbf{Reiknigreind}\\Reikniverkefni 1}
\author{Hörður Freyr Yngvason}
\date{}

\begin{document}
\maketitle


\section{Inngangur}

Í verkefninu skoðum við \emph{back-propagation} á margra-laga tauganet
fyrir flokkunarverkefni. Við höfum gefna útfærslu á slíku reikniriti,
þar sem skekkjan er metin út frá 2-normi fjarlægðar frá réttum gildum, og
viljum breyta kóðanum þannig að notast verði við \emph{cross-entropy}
fyrir skekkjumat.

Til að prófa útfærsluna höfum við síðan safn af 624 svarthvítum
andlitsmyndum \cite{faces} sem unnt er að flokka á ýmsan hátt; t.d. eru 311 myndanna
af viðfangsefni með sólgleraugu. Í seinni hluta verkefnisins reynum
við að læra að flokka myndirnar með tauganeti af þessari gerð.


\section{Fræðileg umgjörð}

Táknum með $h:\mathbb R\to\mathbb R$ þjált einhalla vaxandi fall sem er
takmarkað að ofan og neðan, köllum það \emph{yfirfærslufall}.  Við
byrjum með með fjölskyldu para inntaks- og úttaksgilda $(x_i,t_i)$ og
setjum upp endanlegt stefnt net hnúta $i$ með gildi $a_i$ og $z_i$, og
leggja $ji \equiv i\to j$ með þyngdir $w_{ij}$, sem tengjast með formúlunum
\[
a_j = \sum_{i\to j} w_{ji} z_i
\]
og
\[
z_j = h(a_j),
\]
þar sem $i\to j$ þýðir að við höfum legg með þyngd $w_{ji}$ frá $i$ til
$j$ (getum aldrei farið afturábak). Inntaksgildin $x_i$ eru einu gildin
$z_i$ þannig að ekki er til tenging $j\to i$ og úttaksgildin eru þau
gildi $y_k:=z_k$ þannig að ekki sé til tenging $k\to j$. Netið skiptist
síðan upp í \emph{lög}, sem eru jafngildisflokkar hnútanna m.t.t. fjölda
skrefa frá inntaki, og úttaksgildin $y_k$ þurfa öll að vera í sama
laginu. Við viljum síðan lágmarka gefna skekkju $E(\mathbf w)$ með
tilliti til þyngdanna $w_{ji}$ yfir öll hugsanleg inntaksgildi $x_i$ (en
við látum okkur nægja að lágmarka fyrir úrtakið). 

Nálgunin, sem tekin er í tauganetinu okkar, er að beita \emph{gradient
descent}, til að finna staðbundið lággildi á $E$ m.t.t. $\mathbf w$ þar
sem afleiðurnar $\partial E/\partial w_{ji}$ eru reiknaðar afturábak með
\emph{back-propagation}: Skilgreinum hjálparstærðir 
\[
\delta_j :=  -\frac{\partial E}{\partial a_j}
\]
fyrir alla hnúta $j$. Gerum síðan ráð fyrir
að við getum reiknað gildin $\delta_k$ fyrir alla úttakshnúta $k$.
Af keðjureglunni fæst að 
\[
\frac{\partial E}{\partial w_{ji}}
= \frac{\partial E}{\partial a_j}\frac{\partial a_j}{\partial w_{ji}}
\]
og með því að sjóða saman allt að ofan fæst
\[
\delta_i
= - \frac{\partial E}{\partial a_j}
= - \sum_{i\to j} \frac{\partial E}{\partial a_j} \frac{\partial a_j}{\partial a_i}
= h'(a_i) \sum_{i\to j} w_{ji}\delta_j
\]
sem gefur okkur leið til að meta $\delta_i$ út frá öllum $\delta_j$ þannig að
$i\to j$; þ.e.a.s. við getum metið skekkjuna á hverju lagi út frá
skekkjunni á laginu fyrir ofan (þ.e. laginu sem er nær úttakinu).


\section{Cross-entropy}

Cross-entropy skekkjufallið er gefið með 
\[
E(\mathbf w) 
= - \sum_k (t_k \log(y_k) + (1-t_k) \log(1-y_k)).
\]
Til að stinga þessu falli inn í \emph{back-propagation} reikniritið
nægir okkur nú að reikna frumstillingarnar
\[
\delta_k
= -\frac{\delta E}{\delta a_k};
\]
munum að $y_k = h(a_k)$ svo $\partial y_k / \partial a_k = h'(a_k)$ og
fáum 
\[
\delta_k
= \frac{\partial E}{\partial a_k}
= \frac{t_k}{y_k}h'(a_k)
- \frac{1-t_k}{1-y_k}h'(a_k)
= h'(a_k)\frac{t_k - y_k}{y_k(1-y_k)}
\]
Í okkar tilfelli notum við \emph{logistic sigmoid} fallið
\[
h(x) := \frac{1}{1+e^{-x}}
\]
með 
\[
h'(x) = h(x)(1-h(x))
\]
svo að 
\[
\delta_k 
= t_k - y_k.
\]
Breytt útgáfa af skjalinu \texttt{vbp.m}, sem notar cross-entropy
reiknað á þennan hátt, er í viðauka.


\subsection{XOR-prófið}

Þekkt er að XOR-prófið er ekki línulega sundurgreinanlegt, svo það er
tilvalið í að bera almenn tauganet saman við stakar taugar. Til að
sannreyna að forritið virki enn eftir breytinguna má keyra kóðabútinn
í viðauka \ref{sec:xor-test} á blaðsíðu \pageref{sec:xor-test}; dæmi um
þróun skekkjunnar má sjá á mynd \ref{fig:xor-test}; þar fékkst
útkomuvigurinn 
\begin{verbatim}
   0.0055121   0.9942149   0.9942143   0.0086114
\end{verbatim}
á móti réttu XOR-gildunum
\begin{verbatim}
   0   1   1   0
\end{verbatim}
á inntaksgildin
\begin{verbatim}
  (0,0)  (0,1)  (1,0)  (1,1)
\end{verbatim}

\begin{figure}[h]
  \begin{center}
    \includegraphics[width=\linewidth]{xor_example.pdf}
  \end{center}
  \caption{Skekkjan sem fall af fjölda ítrana í XOR-prófinu}
  \label{fig:xor-test}
\end{figure}


\section{Myndgreining}

Forritin fyrir þennan hluta voru skrifuð að mestu í C, örlítið í Octave
og límt saman í Bash. Til að byrja með flokkum við myndirnar í forritunum með
gildum úr $[0,1]$, þannig að 0 er nei og 1 er já fyrir gefinn flokk.

\subsection{Að þekkja fólk með sólgleraugu}

Munum að við höfum 624 myndir; þar af eru 311 með sólgleraugu og 313 án
sólgleraugna. Við sjáum að fjöldi mynda í hverjum flokki m.t.t.
sunglasses/open skiptingarinnar (tafla \ref{tab:sunglasses} bls.
\pageref{tab:sunglasses}) er nokkuð
jafn (gerum ráð fyrir að notandinn skipti ekki máli). Allt bendir því
til að tauganetið fái jafna möguleika á því að læra hvern flokk.

\begin{table}[h!]
  \centering
  \begin{tabular}{l|ll}
    & sunglasses & open \\\hline\hline
    straight & 78 & 78 \\
    left & 78 & 79 \\
    right & 78 & 77 \\
    up & 77 & 79 \\
    \hline
    neutral & 78 & 80 \\
    happy & 77 & 78 \\
    sad & 78 & 78 \\
    angry & 78 & 77 \\
  \end{tabular}
  \caption{Dreifing á sólgleraugum milli flokka}
  \label{tab:sunglasses}
\end{table}

Þjálfum nú tauganetið við eftirfarandi skilyrði:
\begin{itemize}
  \item Inntak er myndir í lægstu upplausn, hæðin er $30$ punktar og
    breiddin er $32$ punktar.
  \item Úttakið er einn flokkur, þ.e. hvort einstaklingur er með
    sólgleraugu
  \item Eitt falið lag og með tveimur hnútum
  \item 8000 ítranir, lærdómshraði $\eta = 0.1$ og skriðþungi
    $\mu=0.3$.
  \item Þjálfunargögnin eru um 70\% af inntakinu, þ.e. 437 myndir; hinar
    187 eru notaðar í prófun.
\end{itemize}

\begin{figure}[h!]
  \begin{center}
    \includegraphics[width=\linewidth]{sunglasses-error.pdf}
  \end{center}
  \caption{Þjálfunar- (blá) og prófunarskekkjan (rauð) þegar við látum
  netið læra að þekkja flokkana \emph{sunglasses} og \emph{open} með
  eitt tveggja-hnúta millilag, $\eta=0.1$ og $\mu=0.3$; 8000
  umferðir. Skekkjan dettur í lokin niður í um $0.01$.
  Sjáum auk þess að að prófunarskekkjan fer að vaxa upp úr
  1000 ítrunum; það þýðir að netið fer að \emph{oflæra} gögnin upp úr
  þeim fjölda ítrana.}
  \label{fig:sunglasses-error}
\end{figure}
Hnútarnir í millilaginu virðast taka mjög svipuð gildi og
úttakshnúturinn fyrir ekki-sólgleraugu, en úttaksgildin þjappast frekar
upp að endapunktunum 0 og 1 í ákvörðunarbilinu $[0,1]$. Til dæmis fékkst
fyrir millilagið (slembið prófunarúrtak) í sömu keyrslu og þeirri sem
gaf mynd \ref{fig:sunglasses-weights}:
\begin{verbatim}
Node activations:   0.010169 0.008465 
output activations: 0.998648 0.001352 
Target activations: 1.000000 0.000000 

Node activations:   0.014515 0.010832 
output activations: 0.998579 0.001421 
Target activations: 1.000000 0.000000 

Node activations:   0.923234 0.936433 
output activations: 0.000576 0.999425 
Target activations: 0.000000 1.000000 
\end{verbatim}

\begin{figure}[h!]
  \begin{center}
    \includegraphics[width=0.25\linewidth,angle=-90]{sunglasses-w0.pdf}
    \includegraphics[width=0.25\linewidth,angle=-90]{sunglasses-w1.pdf}
  \end{center}
  \caption{Þyngdirnar frá inntaki yfir á millilag (fyrri myndin er fyrri
  línuvigurinn) þegar læra skal flokkana \emph{sunglasses} og
  \emph{open}. Mjög dökkir litir svara til neikvæðra þyngda og ljósir
  litir svara til jákvæðra þyngda, en meðalgráir litir eru líklega
  þyngdir í grennd við 0. Veitum því sérstaklega athygli að sólgleraugun
  fá ekkert frekar jaðarþyngdir (þ.e. hvít/svört gildi) en aðrir punktar; }
  \label{fig:sunglasses-weights}
\end{figure}

Af grafinu í mynd \ref{fig:sunglasses-error} er nokkuð ljóst að við erum
að \emph{oflæra} gögnin. Af þyngdunum á mynd
\ref{fig:sunglasses-weights} sést að netið tekur greinilega tillit til
fleiri atriða en sólgleraugna eingöngu. Prófum þess í stað að hætta
eftir 1000 ítranir. Þróun villunar má þá sjá á mynd
\ref{fig:sunglasses-error-1} og tilheyrandi þyngdir má sjá á mynd
\ref{fig:sunglasses-weights-1}.

\begin{figure}[h!]
  \begin{center}
    \includegraphics[width=\linewidth]{sunglasses-error-1.pdf}
  \end{center}
  \caption{1000 ítranir, $\eta=0.1, \mu=0.3$. Flokkarnir \emph{sunglasses} og
  \emph{open}}
  \label{fig:sunglasses-error-1}
\end{figure}
\begin{figure}[h!]
  \begin{center}
    \includegraphics[width=0.25\linewidth,angle=-90]{sunglasses-w0-1.pdf}
    \includegraphics[width=0.25\linewidth,angle=-90]{sunglasses-w1-1.pdf}
  \end{center}
  \caption{1000 ítranir, $\eta=0.1,\mu=0.3$. Flokkar \emph{sunglasses} og
  \emph{open}. Í ljósi þess að við viljum þekkja
  sólgleraugu, þá er þetta eðlilegra en þegar teknar voru 8000 ítranir
  (ljósir blettir við augun).}
  \label{fig:sunglasses-weights-1}
\end{figure}


\subsection{Að læra alla 10 flokkana}

Reynum nú að láta tauganetið læra alla 10 flokkana í einu. Fyrsta
tilraun verður þannig:
\begin{itemize}
  \item Lægsta upplausn, 30 raðir og 32 dálkar í mynd.
  \item Eitt 10-hnúta millilag.
  \item 8000 ítranir.
  \item $\eta = 0.1$, $\mu=0.3$.
  \item 70\% úrtaks, 437 myndir, til þjálfunar; 30\%, 187 myndir, til
    prófunar.
\end{itemize}
Þróun skekkjunnar má sjá á mynd \ref{fig:all-error} og þyngdirnar á mynd
\ref{fig:all-weights}. Rétt eins og í síðustu undirgrein sjáum við að
við fórum allt of margar ítranir, því prófunarskekkjan staðnar
greinilega (fyrir utan sveiflur) eftir um 1000 ítranir, og þyngdirnar
bera merki þess að reyna að læra of flókin mynstur (t.d. fá axlir vægi,
sem er óeðlilegt).

Prófum nú á móti að fara 1000 ítranir við annars sömu skilyrði.
Niðurstöðurnar eru á myndum \ref{fig:all-error-1} og
\ref{fig:all-weights-1}. 

\begin{figure}[h!]
  \begin{center}
    \includegraphics[width=\linewidth]{all-error.pdf}
  \end{center}
  \caption{8000 ítranir, $\eta=0.1,\mu=0.3$. Allir flokkar. Sjáum að við
  byrjum að oflæra strax upp úr 1000 ítrunum.}
  \label{fig:all-error}
\end{figure}
\begin{figure}[h!]
  \begin{center}
    \begin{tabular}{ccccc}
      \includegraphics[width=0.12\linewidth,angle=-90]{all-w0.pdf}
      & \includegraphics[width=0.12\linewidth,angle=-90]{all-w1.pdf}
      & \includegraphics[width=0.12\linewidth,angle=-90]{all-w2.pdf}
      & \includegraphics[width=0.12\linewidth,angle=-90]{all-w3.pdf}
      & \includegraphics[width=0.12\linewidth,angle=-90]{all-w4.pdf}
      \\
      \includegraphics[width=0.12\linewidth,angle=-90]{all-w5.pdf}
      & \includegraphics[width=0.12\linewidth,angle=-90]{all-w6.pdf}
      & \includegraphics[width=0.12\linewidth,angle=-90]{all-w7.pdf}
      & \includegraphics[width=0.12\linewidth,angle=-90]{all-w8.pdf}
      & \includegraphics[width=0.12\linewidth,angle=-90]{all-w9.pdf}
    \end{tabular}
  \end{center}
  \caption{8000 ítranir, $\eta=0.1,\mu=0.3$. Allir flokkar. Gera má sér í
  hugarlund að myndin efst til vinstri gefi þeim sem horfa til vinstri
  meira vægi, og eins að myndin hægra megin við hana, gefi þeim vægi sem horfa
  til áfram. Mynstrin eru þó frekar óskýr og netið er greinilega að
  oflæra.}

  \label{fig:all-weights}
\end{figure}


\begin{figure}[h!]
  \begin{center}
    \includegraphics[width=\linewidth]{all-error-1.pdf}
  \end{center}
  \caption{1000 ítranir, $\eta=0.1,\mu=0.3$. Allir flokkar.}
  \label{fig:all-error-1}
\end{figure}
\begin{figure}[h!]
  \begin{center}
    \begin{tabular}{ccccc}
        \includegraphics[width=0.12\linewidth,angle=-90]{all-w0-1.pdf}
      & \includegraphics[width=0.12\linewidth,angle=-90]{all-w1-1.pdf}
      & \includegraphics[width=0.12\linewidth,angle=-90]{all-w2-1.pdf}
      & \includegraphics[width=0.12\linewidth,angle=-90]{all-w3-1.pdf}
      & \includegraphics[width=0.12\linewidth,angle=-90]{all-w4-1.pdf}
      \\
        \includegraphics[width=0.12\linewidth,angle=-90]{all-w5-1.pdf}
      & \includegraphics[width=0.12\linewidth,angle=-90]{all-w6-1.pdf}
      & \includegraphics[width=0.12\linewidth,angle=-90]{all-w7-1.pdf}
      & \includegraphics[width=0.12\linewidth,angle=-90]{all-w8-1.pdf}
      & \includegraphics[width=0.12\linewidth,angle=-90]{all-w9-1.pdf}
    \end{tabular}
  \end{center}
  \caption{1000 ítranir, $\eta=0.1,\mu=0.3$. Allir flokkar.}
  \label{fig:all-weights-1}
\end{figure}

\subsection{Túlkun á niðurstöðum}

Með því að skoða þyngdirnar á myndum \ref{fig:sunglasses-error-1} og
\ref{fig:all-error-1} nánar þá sjáum við að netið virðist fyrst og
fremst læra öfgapunkta, þ.e. mjög ljósa og mjög dökka punkta, í myndunum
sem uppfylla jafnframt það skilyrði að vera frekar svipaðir í gegn um
dæmin, með auknu vægi á þá punkta sem hafa fylgni með
réttum-úttaksgildum. Til dæmis er húðliturinn almennt ljós og afmarkar
útlínur höfuðsins, sem er fylgist að með stefnu höfuðs, svo við fáum
almennt eitthvað sem líkist höfði á þyngdirnar. 

Eins, þá eru sólglerugun áberandi dökk miðað við annars frekar ljóst
svæði við gleraugnalaus augu, svo við ættum að fá frekar sterkan mun
milli augnsvæðisins og umhverfisins. Það er vissulega svo á mynd
\ref{fig:sunglasses-error-1}, en sú mynd líður þó fyrir það að
\emph{sólgleraugun færast til með stefnu höfuðsins}. Til að undirstrikra
þessa fylgni, sjá mynd \ref{fig:sunglasses-straight}, sem sýnir hversu
mikið auðveldara verkefnið verður þegar við takmörkum
sólgleraugnaverkefnið við andlit sem snúa beint áfram. Reyndar bendir
þetta líka til þess að við gætum e.t.v. notað sólgleraugnaflokkinn til
að hjálpa netinu að læra hvernig höfuðið á myndinni snýr (því hann kemur
svo sterkur og staðbundinn inn).

Að lokum, athugum að netinu gengur afskaplega illa að læra svipbrigðin á
myndunum; við því var þó að búast vegna þess að upplausnin í myndunum er
lítil, gildin í punktunum með svipbrigðum dreifast yfir stóran hluta
$[0,1]$, og netið lærir óháð röð punkta í myndinni (svo okkur gengur
alveg eins hvernig sem við umröðum punktunum, svo lengi sem breytingni
varðveitist milli mynda).

\begin{figure}[h!]
  \begin{center}
    \includegraphics[width=\linewidth]{sunglasses-straight-error.pdf}
    \\
    \includegraphics[angle=-90,width=0.25\linewidth]{sunglasses-straight-w0.pdf}
    \includegraphics[angle=-90,width=0.25\linewidth]{sunglasses-straight-w1.pdf}
  \end{center}
  \caption{Sólgleraugnaverkefnið þar sem úrtakið er takmarkað við andlit
  sem snúa beint áfram. Netið er fljótara að læra (bara 1000 ítranir að
  komast niður fyrir 0.1 í þjálfunarskekkju) og það gefur eðlilegri þyngdir
  m.t.t. verkefnisins (nánast bara sólgleraugun fá vægi). }
  \label{fig:sunglasses-straight}
\end{figure}

\subsection{Lágmörkun á prófunarskekkju}

Athugum að það, að reyna að tákna myndirnar með gildum úr $[0,1]$
þannig að 0 er nei og 1.0 er já, er ekki endilega ekki. Við höfum
nefnilega einhalla færslufallið $h$ sem uppfyllir $h(x)\to 0$ þegar
$x\to -\infty$ og $h(x)\to 1$ þegar $x\to+\infty$; svo til að komast sem
næst þeim gildum gætum við lent í að klemma gildin í $\mathbf w$ við
$\pm\infty$. Með því að nota annað skekkjumat en cross-entropy kæmi til
greina að láta gildin dreifast á bilið $[0.1,0.9]$, því það eru gildi
sem $h$ getur náð í endanlegum stærðum. Með cross-entropy kæmi það hins
vegar ekki til greina, því við erum háð $h$ í frumstillingu $\delta_k$.
Reyndar vill líka svo til að þyngdirnar okkar verða ekki svo stórar í
dæmunum að ofan (þrátt fyrir hættuna á því). Önnur leið væri að búa til
nýtt skekkjumat 
\[ 
G(\mathbf w ) = \alpha E(\mathbf w) + \beta M(\mathbf w)
\]
þar sem $M(\mathbf w)$ er svokallaður \emph{regularizer}, sem gegnir því
hlutverki að halda þyngdunum í skefjum. Ekki gafst tími til að kanna
þessar hugmyndir nánar.

\subsection{Varðandi önnur gildi á $\eta$, $\mu$ o.fl.}

Fleiri gildi á öllum stikum voru prófuð við gerð verkefnisins en
niðurstöðurnar ekki geymdar og því ekki settar fram hér. Almennt virðist
sem hærri gildi á $\eta$ gefi hraðari samleitni upp að vissu marki, en
örar og háar sveiflur koma fram í bæði þjálfunar- og prófunarskekkju
eftir því sem $\eta$ eykst. Að lokum hættir netið að geta
lært og skekkjan helst föst í upphafsgildi. Þannig er hægt að ná fram
svipaðri samleitni og í 8000 ítrunum fyrir $\eta=0.1,\mu=0.3$ að ofan, í 2000  
ítrunum fyrir svolítið hærri gildi á $\eta$, en samleitnin verður
óstöðug á köflum. Skriðþunginn $\mu$ virðist hjálpa til við samleitni í
tilfellum þegar $\eta$ er mjög lítið og jafnvel geta minnkað
óreiðukenndar sveiflur í skekkju, en að sama skapi getur hann gert illt
verra fyrir stærri $\eta$. Varðandi millilög netsins, þá fengust bestu
niðurstöðurnar með því að hafa eitt millilag með jafnmörgum hnútum og
flokkarnir sjálfir; annars gekk ágætlega að leysa t.d.
sólgleraugnaverkefnið með engu millilagi. Ekki var betra að bæta við
fleiri lögum, en þar kenni ég þó um skorti á \emph{regularizer} fyrir
skekkjuna (fallið $M$ úr seinustu undirgrein); enda hefur tauganet með
fleiri lögum meira \emph{minni}, þ.e. getur lært fleiri og flóknari hluti.

\appendix


\section{\texttt{vbp.m} með cross-entropy}
\verbatiminput{vbp.m}

\section{XOR-prófið}\label{sec:xor-test}
\verbatiminput{test_vbp_xor.m}

\section{VBP útfært í C}
Virkar í GCC-4.4 og ICC-12.0.0 á nýlegu Ubuntu. Þarf útfærslu á BLAS
\cite{blas} til
að keyra.


\subsection{Skilgreiningaskjalið \texttt{hfy\_vbp.h}}
\verbatiminput{hfy_vbp.h}

\subsection{Útfærsluskjalið \texttt{hfy\_vbp.c}}
\verbatiminput{hfy_vbp.c}

\section{Keyrsluforrit fyrir andlitsprófin}
Til að keyra forritin þarf nýlega útgáfu libnetpbm \cite{netpbm} auk
\texttt{hfy\_vbp} ofan. Forritin má t.d. þýða með
\begin{verbatim}
cc -O3 hfy_loadfaces.c hfy_vbp.c -lblas -lnetpbm -o loadfaces 
\end{verbatim}
fyrir gcc, clang o.fl. á móti atlas í ubuntu; eða
\begin{verbatim}
icc -fast hfy_loadfaces.c hfy_vbp.c -mkl -lnetpbm -o loadfaces 
\end{verbatim}
fyrir icc á móti mkl.

\subsection{Útfærsluskjalið \texttt{hfy\_loadfaces.c}}
\verbatiminput{hfy_loadfaces.c}

\begin{thebibliography}{99}
  \bibitem{faces} \url{http://kdd.ics.uci.edu/databases/faces/faces.html}
  \bibitem{blas} \url{http://netlib.org/blas}
  \bibitem{netpbm} \url{http://netpbm.sf.net}
\end{thebibliography}
\end{document}

